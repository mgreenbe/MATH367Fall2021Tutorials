\documentclass{exam}

\usepackage{amsmath, amssymb}
\begin{document}

\title{Math 367 -- Tutorial \#1}
\author{Matthew Greenberg and Keira Gunn}
\date{September 14-16, 2021}
\maketitle

\begin{questions}
    \question Evaluate $z_x$ and $z_y$, where:
    \begin{parts}
        \part $z=xe^{x^2+y^2}$
        \begin{solution}
            \begin{align*}
                z_x &= e^{x^2+y^2} + xe^{x^2+y^2}(2x),\\
                z_y &= xe^{x^2+y^2}(2y)
            \end{align*}
        \end{solution}

        \part $z=\cos(e^{x^2y^3})$
        \begin{solution}
            \begin{align*}
                z_x &= -\sin(e^{x^2y^3})e^{x^2y^3}(2xy^3)\\
                z_y &= -\sin(e^{x^2y^3})e^{x^2y^3}(3x^2y^2)
            \end{align*}
        \end{solution}

        \part $xy^2 + yz^2 + xyz = 1$
        \begin{solution}
            Differentiate both sides with respect to $x$,
            remembering that $z$ is a functino of $x$:
            \[
                y^2 + 2yzz_x + yz + xyz_x = 0
            \]
            Now solve for $z_x$:
            \[
                z_x = -\frac{y^2}{2yz + xy}
            \]
            Differentiate both sides with respect to $y$,
            remembering that $z$ is a functino of $y$:
            \[
                2xy + z^2 + 2yzz_y + xz + xyz_y = 0
            \]
            Now solve for $z_y$:
            \[
                z_t = -\frac{2xy + z^2 + xz}{2yz + xy}
            \]
        \end{solution}

        \part $z=x^y$
        \begin{solution}
            To find $z_x$, just use the power rule:
            \[
                z_x = yx^{y-1}
            \]

            To find $z_y$, take logarithms of both sides of $z=x^y$:
            \[
                \ln z = y\ln x
            \]
            Differentiate both with respect to $y$, remebering that $z$ is a functino of $y$:
            \[
                \frac1z z_y = \ln x
            \]
            Solve for $z_y$:
            \[
                z_y = z\ln x = x^y\ln x
            \]
            
            Alternatively, you could just look up the derivative of $f(x)=a^x$. The above calculation is just a derivation of that formula.
        \end{solution}
    \end{parts}

    \question Let
    \[
        f(\mathbf{x}) = \|\mathbf{x}\|^r,
        \quad \mathbf{x}\in\mathbb{R}^n.
    \]
    Show that
    \[
        \nabla f(\mathbf{x}) = r\|\mathbf{x}\|^{r-2}\mathbf{x}.
    \]

    \begin{solution}
        We have:
        \begin{align*}
            f_{x_i}(\mathbf{x}) &= \frac{\partial}{\partial x_i}
            \left(x_1^2+\cdots x_n^2\right)^{r/2}\\
            &= \frac{r}{2}\left(x_1^2+\cdots x_n^2\right)^{(r-2)/2}2x_i\\
            &= r\|x\|^{r-2}x_i
            \intertext{Therefore,}
            \nabla f(\mathbf{x}) &= (r\|x\|^{r-2}x_1,\ldots,r\|x\|^{r-2}x_n)\\
            &= r\|x\|^{r-2}(x_1,\ldots,x_n)\\
            &= r\|x\|^{r-2}\mathbf{x}.
        \end{align*}
        
    \end{solution}

    \question
    Let $y=f(\mathbf{x})$, $\mathbf{x}\in\mathbb{R}^n$.
    Show that
    \[
        \nabla y^r = ry^{r-1}\nabla y.
    \]

    \begin{solution}
        In class, we noted the following form of the chain rule,
        where $g$ is a real-valued function of a single variable:
        \[
            \nabla g(f(\mathbf{x})) = g'(f(\mathbf{x}))\nabla f(\mathbf{x})
        \]
        If $g(t)=t^r$, then $g'(t)=rt^{r-1}$ and
        \[
            \nabla y^r = \nabla g(y) = g'(y)\nabla y  = r y^{r-1}\nabla y.
        \]

        Alternatively, you can work from first principles:
        \begin{align*}
            \frac{\partial}{\partial x_i}f(\mathbf{x})^r
            &= r f(\mathbf{x})^{r-1} f_{x_i}(\mathbf{x})
            \intertext{Therefore,}
            \nabla f(\mathbf{x})^r &= (r f(\mathbf{x})^{r-1} f_{x_1}(\mathbf{x}),\ldots,r f(\mathbf{x})^{r-1} f_{x_n}(\mathbf{x}))\\
            &= rf(x)^{r-1}(f_{x_1}(\mathbf{x}),\ldots,f_{x_n}(\mathbf{x}))\\
            &= rf(x)^{r-1}\nabla f(\mathbf{x})
            \intertext{Equivalently,}
            \nabla y^r &= r y^{r-1} \nabla y.
        \end{align*}
    \end{solution}
\end{questions}


\end{document}