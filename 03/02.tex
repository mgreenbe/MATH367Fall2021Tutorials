\documentclass{exam}

\usepackage{amsmath, amssymb}

\newcommand{\vu}{\mathbf{u}}
\newcommand{\vzero}{\mathbf{0}}
\begin{document}

\title{Math 367 -- Tutorial \#2}
\author{Matthew Greenberg and Keira Gunn}
\date{September 28-30, 2021}
\maketitle

\begin{questions}
    \setlength\itemsep{1em}
    \question Find the directional derivative of
    \[
        f(x,y) = xe^y + ye^x
    \]
    at $(0,0)$ in the direction making an angle $\theta=\pi/6$ to the horizontal.

    \begin{solution}
        To find directional derivatives, we take dot products with the gradient vector.
        \begin{align*}
            \nabla f(x,y) &= (e^y + ye^x, xe^y + e^x)\\
            \nabla f(0,0) &= (1,1)
        \end{align*}
    The unit vector at an angle of $\pi/6$ to the horizontal is
    \[
        \vu = \left(\frac{\sqrt3}2,\frac{1}{2}\right).
    \]
    We have:
    \[
        D_\vu f(0,0) = \nabla f(0,0)\cdot \vu = \frac{\sqrt3 + 1}{2}
    \]
    \end{solution}

    \question
    Find and classify the critical points of the function.

    \begin{parts}
        \setlength\itemsep{0.5em}
        \part $f(x,y) = x^2 + xy + y^2 - 6x + 6$

        \begin{solution}
            To find the critical points, we solve $\nabla f = (0,0)$.
            \[
                \nabla f(x,y) = (2x + y - 6, x + 2y)
            \]
            The system
            \[
                2x+y-6=0,\quad x+2y=0
            \]
            has unique solution
            \[
                (x,y)=(4,-2).
            \]
            To classify the critical point at $(4,-2)$, we compute second partials:
            \[
                f_{xx}=2,\quad f_{xy} = 1,\quad f_{yy} = 2
            \]
            The associated discriminant quantity is
            \[
                D = f_{xx}f_{yy} - f_{xy}^2 = (2)(2) - 1 = 3 > 0.
            \]
            Therefore, by the second derivative test, $f(x,y)$ has a local minimum at $(0,0)$.
            
        \end{solution}
        \part $f(x,y) =x^3 + y^2 + 2xy -4x - 3y + 5$

        \begin{solution}
            To find the critical points, we solve $\nabla f = (0,0)$.
            
            \[
                \nabla f(x,y) = (3x^2 + 2y - 4, 2y + 2x - 3)
            \]
            To solve the system
            \[
                3x^2 + 2y - 4=0,\quad  2y + 2x - 3=0,
            \]
            we solve the second equation for $2y$,
            \[
                2y = 3-2x,
            \]
            and substitute into the first:
            \begin{align*}
                0 &= 3x^2 + (3-2x) - 4\\
                &= 3x^2 -2x - 1\\
                &= (3x + 1)(x - 1)\\
            \end{align*}
            We get
            \[
            x = -\frac{1}{3},\quad 1
            \]
            Computing the corresponding $y$-values, we get that the critical points of $f$ are at
            \[
                P=\left(-\frac13,\frac{11}6\right),\quad Q=\left(1,\frac12\right).
            \]

            To classify these critical points, we compute second derivatives:
            \[
                f_{xx}(x,y) = 6x,\quad f_{xy}(x,y) = 2,\quad f_{yy}(x,y) = 2.
            \]
            As
            \[
                D(P) = 6\left(-\frac13\right)(2)-2^2 = -8 <0
            \]
            $f$ has a saddle point at $P$.
            Since
            \[
                D(Q) = 6(1)(2) - 2^2 = 8 > 0 
            \]
            and $f_{xx}(Q)=6>0$, $f$ has a local minimum at $Q$.
        \end{solution}
        
        \part $f(x,y,z) = \frac12(5x^2 + 11y^2 + 2z^2 + 16xy + 20xz -4yz)$

        \begin{solution}
            Let's write $f$ in matrix form:
            \[
                f(x,y,z) = \begin{pmatrix}
                    x&y&z
                \end{pmatrix}A\begin{pmatrix}
                    x\\y\\z
                \end{pmatrix},\quad
                A = \begin{pmatrix}
                    5 & 8 & 10\\8&11&-2\\10&-2&2
                \end{pmatrix}
            \]
            We have:
            \[
                \nabla f(x,y,z) = A\begin{pmatrix}x\\y\\z
                \end{pmatrix}
            \]
            Since $A$ is invertible (its determinant is $-1458\neq 0$),
            \[
                \nabla f(x,y,z)=(0,0,0)
                \Longleftrightarrow
                (x,y,z)=(0,0,0).
            \]
            Thus, $(0,0,0)$ is the only critical point of $f$.
            The matrix $A$ has eigenvalues $-9$, $9$, and $18$.
            Since there are both positive and negative numbers among these,
            we conclude that $(0,0,0)$ is a saddle point of $f$.
        \end{solution}
    \end{parts}
\end{questions}


\end{document}